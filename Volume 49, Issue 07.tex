%%%%%%%%%%%%%%%%%%%%%%%%%%%%%%%%%%%%%%%%%
% Professional Newsletter Template
% LaTeX Template
% Version 1.0 (09/03/14)
%
% Created by:
% Bob Kerstetter (https://www.tug.org/texshowcase/) and extensively modified by:
% Vel (vel@latextemplates.com)
% 
% This template has been downloaded from:
% http://www.LaTeXTemplates.com
%
% License:
% CC BY-NC-SA 3.0 (http://creativecommons.org/licenses/by-nc-sa/3.0/)
%
%%%%%%%%%%%%%%%%%%%%%%%%%%%%%%%%%%%%%%%%%

\documentclass[9pt]{extarticle} % The default font size is 10pt; 11pt and 12pt are alternatives

\input{structure.tex} % Include the document which specifies all packages and structural customizations for this template

\begin{document}

%--------------------------------------------------------------------------------
% HEADER DETAILS
%--------------------------------------------------------------------------------

\pagestyle{fancy}
\fancyhf{}
\chead{segfault@csh.rit.edu}
\rhead{\today}
\lhead{Volume XLIX Issue \#7}
\addtolength\footskip{-15px}
\cfoot{"I'm about to drop some science on your bitch ass." ~Jimmy DiGrazia}

%----------------------------------------------------------------------------------------
%	HEADER IMAGE
%----------------------------------------------------------------------------------------

\begin{figure}[H]
\centering\vspace{0.5cm}\includegraphics[width=0.8\linewidth]{imgs/segfault.png}
\end{figure}

%--------------------------------------------------------------------------------
% HEADER QUOTE
%--------------------------------------------------------------------------------

\vspace{-15px}
\begin{quote}
\centering
\textbf{\textit{Freshly Hung}}
\end{quote}
\vspace{10px}

%----------------------------------------------------------------------------------------
%	SIDEBAR - FIRST PAGE
%----------------------------------------------------------------------------------------

\vspace{-0.5cm}\begin{minipage}[t]{.35\linewidth} % Mini page taking up 35% of the actual page
\begin{mdframed}[style=sidebar,frametitle={}] % Sidebar box

%-----------------------------------------------------------

\hypertarget{contents}{\textbf{{\large This week on floor\ldots}}} % \hypertarget provides a label to reference using \hyperlink{label}{link text}
\begin{itemize}
\item \hyperlink{firstnews}{10 Linux Commands}
\end{itemize}

\centerline {\rule{.75\linewidth}{.25pt}} % Horizontal line

%-----------------------------------------------------------

\textbf{Notable Upcoming Events:}
\begin{enumerate}[leftmargin=0.2cm]
\item \textbf{Dockerizing Apps with Steven} 4pm, Monday \\
	Hey freshmen! You know you need those Technical Seminars right? Tying a tie doesn't count!
\\
\item \textbf{Turkey with Efe} 7pm Tuesday\\
	A quick seminar on cooking a turkey by Efe! 
\\
\item \textbf{Soapy Seminar with Drew} 6pm, Wednesday \\
	Learn how to make music come out a shower head! With an RFID tag no less!
\\
\item \textbf{History Discussion} 9pm, Wednesday \\
	Game Jam post mortem!
\\
\item \textbf{Homecoming Hockey Game!} 7pm, Saturday\\
	HOCKEY!!! Let's beat the snot out of em!
\\
\item \textbf{The Pacific with Trevor and Braden} 9:30pm-12am, Sunday\\
	AMAZING War mini-series. Ask Trevor and Braden for more info! 
\\
\end{enumerate}

%-----------------------------------------------------------

%-----------------------------------------------------------

\end{mdframed}
\end{minipage}\hfill % End the sidebar mini page 
%
%----------------------------------------------------------------------------------------
%	MAIN BODY - FIRST PAGE
%----------------------------------------------------------------------------------------
%
\begin{minipage}[t]{.61\linewidth} % Mini page taking up 61% of the actual page
\vspace{-0.4cm}
\hypertarget{firstnews}{\heading{Liam's Top Ten Linux Commands}{6pt}}
 
 \begin{enumerate}[leftmargin=0.2cm]
 	\item \textbf{mupdf} \\
 	mupdf is a lightweight X11/GL PDF Viewing application. It allows for use of Vim keybindings to navigate PDF files. \\
 	Example: mupdf SHENZHEN\textunderscore IO\textunderscore Manual.pdf 
	\item \textbf{grep} \\
 	grep allows you to select lines that match a pattern. Normally you'll pipe something to grep's stdin rather than have grep read directly from a file. \\ 
 	Example: cat /var/log/Xorg.0.log | grep EE
 	\item \textbf{scp}\\
 	scp allows you to securely copy files to and from remote locations.//
 	Example: scp loothelion@filer.csh.rit.edu:.html\textunderscore pages/thelinuxcommandline.pdf
 	\item \textbf{find}\\
 	find searches for files in a directory based on user defined patterns. \\
 	Example: find /users/u24/ram -type f -name *.torrent
 	\item \textbf{nmap}\\
 	nmap is a network exploration tool. \\
 	Example: nmap -PN hagrid.csh.rit.edu
 	\item \textbf{curl} \\
 	curl preforms an web request to "transfer a URL." \\
 	Example: curl http://icanhazip.com 
 	\item \textbf{cut}\\
 	cut removes sections from each line of files. Normally you'll pipe something to cut's stdin rather than run cut directly. \\
 	Example: history | cut -d' ' -f4 
 	\item \textbf{strace}\\
 	strace prints out the system calls and signals made by a program. \\
 	Example: strace echo "Test strace" 
 	\item \textbf{htop}\\
 	htop is an interactive process viewer. \\
 	Example: htop
 	\item \textbf{cacaview}\\
 	cacaview is an ASCII image browser.\\
 	Example: cacaview james-face.jpg\\
 \end{enumerate}

%-----------------------------------------------------------


\end{minipage} % End the main body - first page mini page

\centering\vspace{0.5cm}\includegraphics[width=0.78\linewidth]{imgs/decline.png}
\end{document} 
