%%%%%%%%%%%%%%%%%%%%%%%%%%%%%%%%%%%%%%%%%
% Professional Newsletter Template
% LaTeX Template
% Version 1.0 (09/03/14)
%
% Created by:
% Bob Kerstetter (https://www.tug.org/texshowcase/) and extensively modified by:
% Vel (vel@latextemplates.com)
% 
% This template has been downloaded from:
% http://www.LaTeXTemplates.com
%
% License:
% CC BY-NC-SA 3.0 (http://creativecommons.org/licenses/by-nc-sa/3.0/)
%
%%%%%%%%%%%%%%%%%%%%%%%%%%%%%%%%%%%%%%%%%

\documentclass[9pt]{extarticle} % The default font size is 10pt; 11pt and 12pt are alternatives

\input{structure.tex} % Include the document which specifies all packages and structural customizations for this template

\begin{document}

%--------------------------------------------------------------------------------
% HEADER DETAILS
%--------------------------------------------------------------------------------

\pagestyle{fancy}
\fancyhf{}
\chead{segfault@csh.rit.edu}
\rhead{\today}
\lhead{Volume XLVII Issue \#10}
\addtolength\footskip{-15px}
\cfoot{"My life would be a whole lot easier if I was gay" - Adrian (asvenson)}

%----------------------------------------------------------------------------------------
%	HEADER IMAGE
%----------------------------------------------------------------------------------------

\begin{figure}[H]
\centering\vspace{0.5cm}\includegraphics[width=0.8\linewidth]{segfault.png}
\end{figure}

%--------------------------------------------------------------------------------
% HEADER QUOTE
%--------------------------------------------------------------------------------

\vspace{-15px}
\begin{quote}
\centering
\textbf{\textit{Have fun, don't die}}
\end{quote}
\vspace{10px}

%----------------------------------------------------------------------------------------
%	SIDEBAR - FIRST PAGE
%----------------------------------------------------------------------------------------

\vspace{-0.5cm}\begin{minipage}[t]{.32\linewidth} % Mini page taking up 35% of the actual page
\begin{mdframed}[style=sidebar,frametitle={}] % Sidebar box

%-----------------------------------------------------------

\hypertarget{contents}{\textbf{{\large This week on floor\ldots}}} % \hypertarget provides a label to reference using \hyperlink{label}{link text}
\begin{itemize}
\item \hyperlink{firstnews}{Big-O Notation}
\end{itemize}

%-----------------------------------------------------------

\centerline {\rule{.75\linewidth}{.25pt}} % Horizontal line

\textbf{Evaluations Nominations } \\
Harlan has stepped down as the Evaluations Director due to him going on
co-op next semester. What this means is that we need to start the election
process for the new Evaluations Director. If you know of a person who would
be a good fit for the position, make sure that you submit their name into the
nomination box in the library before E-Board on Thursday. \\

\centerline {\rule{.75\linewidth}{.25pt}} % Horizontal line

%-----------------------------------------------------------

\textbf{Notable Upcoming Events:}
\begin{enumerate}[leftmargin=0.2cm]
\item \textbf{Cryptography Seminar} Dec. 8, 7pm\\
	Cryptography Seminar Part 2!!
\\
\item \textbf{Weird Movie Night} Dec. 8, 9:30pm \\
	This week's movie is Moonwalker!!
\\
\item \textbf{GDC Talks} Dec. 9, 10pm \\
	Come join Liam to watch previous years' GDC talks. Let Liam know
	if there is a particular talk you would want to watch.
\\
\item \textbf{Potter Networking} Dec. 10, 7pm \\
	Come learn more networking with Potter.
\\
\item \textbf{Guitar Hero} Dec. 11, 10pm \\
	Social is hosting a Guitar Hero competition, come show
	off your mad skills and earn prizes!
\\
\item \textbf{Holiday Dinner} Dec. 12, 6:30pm \\
	Holiday dinner is at Burgundy Basin this year. Remember
	to dress up. There is a cash bar there for all of you
	over 21.
\\
\item \textbf{Secret Santa} Dec. 12, 9pm \\
	CSH gift exchange, come watch people cry as they open their gifts.
	We will be doing group photos beforehand so make sure you are there
	early, even if you do not go Holiday Dinner.
\\
\item \textbf{SIH Broomball} Dec. 12, 11pm \\
	All of the Special Interest Houses are having a broomball competition,
	come show off that we truly are the best SIT broomball team!!
\\
\item \textbf{Culture Movie Night} Dec. 13, 8:30pm \\
	The movie this week is Seven Samurai, come relax before we all
	start stressing about finals. \\ 
\end{enumerate}

%-----------------------------------------------------------

%-----------------------------------------------------------

%-----------------------------------------------------------

\end{mdframed}
\end{minipage}\hfill % End the sidebar mini page 
%
%----------------------------------------------------------------------------------------
%	MAIN BODY - FIRST PAGE
%----------------------------------------------------------------------------------------
%
\begin{minipage}[t]{.64\linewidth} % Mini page taking up 61% of the actual page
\vspace{-0.4cm}
\hypertarget{firstnews}{\heading{Big-O Notation}{6pt}}

Big O is the way of measuring the efficiency of an algorithm and how well it 
scales based on the size of the dataset. Imagine you have a list of 10 objects,
and you want to sort them in order. There’s a whole bunch of algorithms you can 
use to make that happen, but not all algorithms are created equal. Some are 
quicker than others, but more importantly is that the speed of an algorithm can
vary depending on how many items it is dealing with. \\
\\
Big O is represented using something like O(n). The O simply denotes that we are 
talking about big O and you can ignore it. The n is the thing the complexity is
in relation to; for programming interview questions this is almost always the 
size of a collection. The complexity will increase or decrease in accordance 
with the size of the data store. Below is a list of the Big O complexities 
in order of how well they scale relative to the dataset.

\textbf{O(1) / Constant Complexity:} 
This means that it will always take the same amount of time, no matter the size
of the input. 1 item takes 1 second, 10 items takes 1 second, 100 items 
takes 1 second. It always takes the same amount of time. This is seen in 
retrieval from arrays or dictionaries, peeking at a queue, or getting the size 
of an array.

\textbf{O(log n) / Logarithmic Complexity:}
Not as good as constant, but still pretty good. The time taken increases with 
the size of the data set, but not proportionately so. This means the algorithm
takes longer per item on smaller datasets relative to larger ones. 1 item 
takes 1 second, 10 items takes 2 seconds, 100 items takes 3 seconds. If your
dataset has 10 items, each item causes 0.2 seconds latency. If your dataset has
100, it only takes 0.03 seconds extra per item. This makes log n algorithms 
very scalable. This is seen in something like a binary tree search since it cuts
half of the items to search at every step ($log_2$).

\textbf{O(n) / Linear Complexity:}
The larger the data set, the time taken grows proportionately. 1 item takes 1
second, 10 items takes 10 seconds, 100 items takes 100 seconds. This is seen
in algorithms like searching a linked list or summing up all the values in an 
array. These algorithms are O(n) since every item needs to be checked. 
This will not scale if your data becomes too large.

\textbf{O(n log n):}
A nice combination of the previous two. Normally there are 2 parts to the sort,
the first loop is O(n), the second is O(log n), which combines to form O(n log n).
This complexity is common in most major searching algorithms, like quicksort and
mergesort.

\textbf{O($n^2$) / Quadratic Complexity:}
Things are getting extra slow. 1 item takes 1 second, 10 items takes 100, 
100 items takes 10000. An example of this is bubble sort or generating every 
combination of your input data. Always try to avoid this.

\textbf{O($2^n$) / Exponential Growth:}
The algorithm takes twice as long for every new element added. 1 item takes 1
second, 10 items takes 1024 seconds, 100 items takes 
1267650600228229401496703205376 seconds. HOLY FUCK AVOID THIS.

\textbf{Some hints on how to figure out Big O for an algorithm:}
\begin{enumerate}
\item Does it have to go through the entire list? There will be an O(n) in 
there somewhere.
\item Does the algorithm's processing time increase at a slower rate than 
the size of the data set? Then there is probably a O(log n) in there.
\item Are there nested loops?  You are probably looking at O($n^2$) or O($n^3$).
\item Is access time constant irrelevant of the size of the dataset? O(1)
\end{enumerate}

%-----------------------------------------------------------

\end{minipage} % End the main body - first page mini page

\end{document} 