%%%%%%%%%%%%%%%%%%%%%%%%%%%%%%%%%%%%%%%%%
% Professional Newsletter Template
% LaTeX Template
% Version 1.0 (09/03/14)
%
% Created by:
% Bob Kerstetter (https://www.tug.org/texshowcase/) and extensively modified by:
% Vel (vel@latextemplates.com)
% 
% This template has been downloaded from:
% http://www.LaTeXTemplates.com
%
% License:
% CC BY-NC-SA 3.0 (http://creativecommons.org/licenses/by-nc-sa/3.0/)
%
%%%%%%%%%%%%%%%%%%%%%%%%%%%%%%%%%%%%%%%%%

\documentclass[9pt]{extarticle} % The default font size is 10pt; 11pt and 12pt are alternatives
\usepackage[procnames]{listings}
\usepackage{color}
\usepackage{makecell}

\definecolor{keywords}{RGB}{255,0,90}
\definecolor{comments}{RGB}{0,0,113}
\definecolor{red}{RGB}{160,0,0}
\definecolor{green}{RGB}{0,150,0}
 
\lstset{language=Python, 
        basicstyle=\ttfamily\small, 
        keywordstyle=\color{keywords},
        commentstyle=\color{comments},
        showstringspaces=false,
        procnamekeys={def,class}}

\input{structure.tex} % Include the document which specifies all packages and structural customizations for this template

\begin{document}

%--------------------------------------------------------------------------------
% HEADER DETAILS
%--------------------------------------------------------------------------------

\pagestyle{fancy}
\fancyhf{}
\chead{segfault@csh.rit.edu}
\rhead{\today}
\lhead{Volume XLVII Issue \#11}
\addtolength\footskip{-15px}
\cfoot{"I do not believe in straight people." - Marc Billow (mbillow)}

%----------------------------------------------------------------------------------------
%	HEADER IMAGE
%----------------------------------------------------------------------------------------

\begin{figure}[H]
\centering\vspace{0.5cm}\includegraphics[width=0.8\linewidth]{imgs/segfault.png}
\end{figure}

%--------------------------------------------------------------------------------
% HEADER QUOTE
%--------------------------------------------------------------------------------

\vspace{-15px}
\begin{quote}
\centering
\textbf{\textit{I mean what is really the worst that can happen?}}
\end{quote}
\vspace{10px}

%----------------------------------------------------------------------------------------
%	SIDEBAR - FIRST PAGE
%----------------------------------------------------------------------------------------

\vspace{-0.5cm}\begin{minipage}[t]{.30\linewidth} % Mini page taking up 30% of the actual page
\begin{mdframed}[style=sidebar,frametitle={}] % Sidebar box

%-----------------------------------------------------------

\hypertarget{contents}{\textbf{{\large This week on floor\ldots}}} % \hypertarget provides a label to reference using \hyperlink{label}{link text}
\begin{itemize}
\item \hyperlink{firstnews}{Secret Santa Round Up}
\item \hyperlink{secondnews}{Containers with Dgonyeo}
\item \hyperlink{thirdnews}{Poll finds little room for compromise on what 
we want in next evaluations director}
\end{itemize}

\centerline {\rule{.75\linewidth}{.25pt}} % Horizontal line

Evaluations elections was this last week and this is your reminder to let
you know that you need to submit your ballot before the election period
closes. Think carefully about your vote as this will affect CSH greatly,
no pressure.

\centerline {\rule{.75\linewidth}{.25pt}} % Horizontal line

%-----------------------------------------------------------

\textbf{Notable Upcoming Events:} \\

\item \textbf{FINALS}: ALL WEEK LONG \\
	You know you should have started studying yesterday

\item \textbf{Open Source Swift}: Dec. 15, 6:30pm \\
	
\item \textbf{Jersey Christmas}: Dec. 15 9:30pm \\
	Come learn what Christmas is like in Jersey with Jackie and
	Schuyler!

\item \textbf{Graphic Seminar}: Dec 16, 7pm \\
	Gerard is going to be on-floor sometime this week and is going to be
	giving a seminar to all of you. Come learn some cool stuff and see
	some Gerard shit.
	
\item \textbf{Winter Break}: Dec 19 - Jan 24 \\
	Get out of this hell hole we call RIT and go visit family and loved 
	ones.

%-----------------------------------------------------------


%-----------------------------------------------------------

%-----------------------------------------------------------

\end{mdframed}
\end{minipage}\hfill % End the sidebar mini page 
%
%----------------------------------------------------------------------------------------
%	MAIN BODY - FIRST PAGE
%----------------------------------------------------------------------------------------
%
\begin{minipage}[t]{.66\linewidth} % Mini page taking up 66% of the actual page
\vspace{-0.4cm}

%-----------------------------------------------------------

\hypertarget{firstnews}{\heading{Secret Santa Round Up}{6pt}} 

Last night was the Computer Science House Holiday Dinner and Secret Santa. A big
thanks to Colin O’Neill for running such entertaining events! Here are some of 
Segfault’s favorite gifts from last night. 

\begin{enumerate}
\item \textbf{Marc Billow} received a beautifully written letter from slack-will
telling him that he knows about Marc's secret lust for JD. The letter provided 
tools to seduce Paul Ugolini instead, kinky. 

\item \textbf{Travis Whitaker} and \textbf{Luke Shadler} both received the same 
poster of Travis crying while Monica Rizzo and Luke both stand together in a photo
the shape of a heart. 

\item A framed photo of \textbf{Nick Mosher} and \textbf{George Morgan} holding
themselves passionately in field of flowers, notice the hand action going on there.

\item \textbf{Sheckle} received more of a mission than a present. She removed 
bottles of ketchup from 5 different boxes with the goal of not covering herself
in ketchup.  

\item \textbf{Brandon Chu} got a care package from home, reminding him of what he
is missing. Oh man that MSG looks appetizing!

\end{enumerate}

%-----------------------------------------------------------

\hypertarget{secondnews}{\heading{Containers 101}{6pt}}
\begin{quote}
\centering
\textbf{\textit{author: Derek Gonyeo (dgonyeo)}}
\end{quote}
	
In order for me to describe what containers are, let's first establish what
virtual machines are, and why they exist. Before any of this container or VM
stuff came about, you would just run applications straight on your server. 
If Sally wanted PHP 5.6 on the server to run her stuff, and Bob wanted PHP 5.1
to run something else, they'd either need to complicate their server by having multiple versions of PHP available, or Bob would need to run his stuff on a different machine.

This is annoying, and it's a problem that VMs can solve. Instead of getting a
whole new computer to run his shitty little website, we have software built to
help you put a secondary OS on the same machine through lies. The new OS
\textit{thinks} it's on its own hardware, but all of the hardware bits that
it is twiddling are secretly just emulated by the original OS. The virtual
machine's kernel goes to send a packet to the network interface? Nah, it's 
just a thing in software that goes to the host OS. The kernel tries to write a
file? More lies. That's not a real hard drive, but just a file on the host OS. 
Now Sally can run her new and shiny PHP app, and Bob can run his old and shitty
PHP app, and each of them get their own web server with only the version of PHP
that they want.

Many companies have built fortunes using lies - err, VMs - to be able to give
each little service / application its own machine. One annoying thing about
this strategy though is it can take a significant amount of work to set up a
VM. If you've ever installed Linux, imagine going through that process manually
for each set of things you want to run. There is tooling around doing this
automagically for you, but if you're not paying Amazon you'll be setting that
up yourself.

In our hypothetical scenario where Bob insists on using a version of PHP from
2005 for some god awful reason, what if we could streamline the process by
using significantly less lies? Ultimately the end goal is just having multiple
web servers with different versions of PHP running on the same machine. Do we
really need to emulate network cards, CPUs, hard drives, and more for each web
server? What if we could only lie about the versions of PHP available on the
system, and let the different web servers use the real block devices and memory
space on the host? Containers are, in a sense, exactly that. Containers are kinda
like VMs in that they're also built with the goal of lying to parts of your computer,
but they aim to be easier to use/setup by lying about significantly less.


\end{minipage} % End the main body - first page mini page

%-----------------------------------------------------------
\newpage

%----------------------------------------------------------------------------------------
%	HEADER IMAGE
%----------------------------------------------------------------------------------------

\begin{figure}[H]
\centering\vspace{0.5cm}\includegraphics[width=0.8\linewidth]{imgs/segfault.png}
\end{figure}

\hypertarget{Thirdnews}{\heading{Poll finds little room for compromise on what 
we want in next evaluations director}{6pt}}

CSHers no longer differ only on ideology. When it comes to considering the next president, the Illuminati and the trumpers seem to disagree on, well, just about everything. \\

A recent Wiki poll by Braden Bowdish finds voters in the two parties express conflicting perspectives on the most basic questions facing the organization. Members of the Illuminati are inclined to think the house is headed in the right direction; members of the other party overwhelmingly say the house has gotten off track. The Illuminati want experience; the Trumpers crave an outsider. And they have different expectations on how much any candidate will be able to do about the organization's biggest problems. \\

The divisions — one factor behind an increasingly frustrated and frustrating political system — are helping to shape the latest election debate. But even a decisive win in the next election by one side or the other is no guarantee they'll be bridged. In the national survey, the Illuminati are more likely than not to say the organization is headed in the right direction, 41\% to 36\%. But by an overwhelming 86\% of Trumpers say the house has gotten off on the wrong track. The poll of 40 likely voters, taken Dec. 2-6, has a margin of error of +/-3 percentage points. \\

Beyond differences on what policies the next evaluation director should advocate, Trumpers and Illuminati diverge even on the personal backgrounds they value and the political tactics they endorse. Three of four Illuminati agree with this statement: "The problems the Computer Science House faces are so serious that it's important to elect an evaluations director who has experience in e-board to address them." That attitude is a good fit with Victoria, who holds a decisive lead for the Illuminati nomination in the poll. "She's got a little more experience and been-there-done-that type of deal," Jacy Hollander, 19, part of the L dwellers, says approvingly. \\

In one of the starkest contrasts in the survey, however, most Trumpers agree instead with this statement: "The problems the Computer Science House faces are so serious that it's time to elect an outsider as evaluations director who can bring a fresh perspective to address them." Is it any surprise that the rich asshole, who has never run for office before, is leading the Trumper field? \\

CSHers by more than 2-1 say they are looking for a new director who would "be willing to compromise to get some things done." Illuminati by close to 3-1 feel that way. But more than a third of Trumpers say the next evaluations director should "stand firm on principle, even if it means he or she can't get some things done." Very conservative voters — an influential force in key Trumper primaries — by 48\%-43\% are more inclined to back standing on principle over compromise. That helps explain the defiant stance Max and others have taken on E-Board that have led to stalemate and shutdown. \\

"It's a lot of bickering, a lot of dysfunction," says Meghan Good, 19, a computer security major and frequent watcher of Gravity Falls. She voted for Trumper Harlan Haskins in spring 2015 but the winter 2016 contender who appealed most to her has been Victoria. Meghan says "restoring e-board" is one of his top priorities for the next evaluations director. \\

On an unseasonably warm day for December, student are hanging around Rochester Institute of Technology's Computer Science House. Regardless of party affiliation, though, the attitude of many toward the evaluations director prospects is more wary than warm. Jame Forcier, 20, who works as the secretary (aka bitch) for e-board is waiting to that he can leave for home. He voted for Illuminati candidate in the last two elections, but he hasn't made up his mind about this election. "They're much more concerned about battling each other than talking about any issues," he says of the evaluations director field. "I'm just looking for a candidate who is concerned with our people and their needs."


\end{document} 
