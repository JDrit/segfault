%%%%%%%%%%%%%%%%%%%%%%%%%%%%%%%%%%%%%%%%%
% Professional Newsletter Template
% LaTeX Template
% Version 1.0 (09/03/14)
%
% Created by:
% Bob Kerstetter (https://www.tug.org/texshowcase/) and extensively modified by:
% Vel (vel@latextemplates.com)
% 
% This template has been downloaded from:
% http://www.LaTeXTemplates.com
%
% License:
% CC BY-NC-SA 3.0 (http://creativecommons.org/licenses/by-nc-sa/3.0/)
%
%%%%%%%%%%%%%%%%%%%%%%%%%%%%%%%%%%%%%%%%%

\documentclass[9pt]{extarticle} % The default font size is 10pt; 11pt and 12pt are alternatives
\usepackage[procnames]{listings}
\usepackage{color}
\usepackage{makecell}

\definecolor{keywords}{RGB}{255,0,90}
\definecolor{comments}{RGB}{0,0,113}
\definecolor{red}{RGB}{160,0,0}
\definecolor{green}{RGB}{0,150,0}
 
\lstset{language=Python, 
        basicstyle=\ttfamily\small, 
        keywordstyle=\color{keywords},
        commentstyle=\color{comments},
        showstringspaces=false,
        procnamekeys={def,class}}

\input{structure.tex} % Include the document which specifies all packages and structural customizations for this template

\begin{document}

%--------------------------------------------------------------------------------
% HEADER DETAILS
%--------------------------------------------------------------------------------

\pagestyle{fancy}
\fancyhf{}
\chead{segfault@csh.rit.edu}
\rhead{\today}
\lhead{Volume XLVIII Issue \#2}
\addtolength\footskip{-15px}
\cfoot{"Marc I just want to say I like dick in my mouth twice as much as you do" - Drew Gottlieb (dag10)}

%----------------------------------------------------------------------------------------
%	HEADER IMAGE
%----------------------------------------------------------------------------------------

\begin{figure}[H]
\centering\vspace{0.5cm}\includegraphics[width=0.8\linewidth]{segfault.png}
\end{figure}

%--------------------------------------------------------------------------------
% HEADER QUOTE
%--------------------------------------------------------------------------------

\vspace{-15px}
\begin{quote}
\centering
\textbf{\textit{I mean what is really the worst that can happen?}}
\end{quote}
\vspace{10px}

%----------------------------------------------------------------------------------------
%	SIDEBAR - FIRST PAGE
%----------------------------------------------------------------------------------------

\vspace{-0.5cm}\begin{minipage}[t]{.30\linewidth} % Mini page taking up 30% of the actual page
\begin{mdframed}[style=sidebar,frametitle={}] % Sidebar box

%-----------------------------------------------------------

\hypertarget{contents}{\textbf{{\large This week on floor\ldots}}} % \hypertarget provides a label to reference using \hyperlink{label}{link text}
\begin{itemize}
\item \hyperlink{firstnews}{Containers with Dgonyeo}
\end{itemize}

\centerline {\rule{.75\linewidth}{.25pt}} % Horizontal line

%-----------------------------------------------------------

\textbf{Notable Upcoming Events:} \\
\item \textbf{<EVENT NAME>} <DATE> \\
	<DESC>
	

%-----------------------------------------------------------


%-----------------------------------------------------------

%-----------------------------------------------------------

\end{mdframed}
\end{minipage}\hfill % End the sidebar mini page 
%
%----------------------------------------------------------------------------------------
%	MAIN BODY - FIRST PAGE
%----------------------------------------------------------------------------------------
%
\begin{minipage}[t]{.66\linewidth} % Mini page taking up 66% of the actual page
\vspace{-0.4cm}
\hypertarget{firstnews}{\heading{Containers 101}{6pt}}
\begin{quote}
\centering
\textbf{\textit{author: Derek Gonyeo (dgonyeo)}}
\end{quote}
	
Containers are a really cool, useful technology that are good for solving a lot
of common problems. One problem though is that they have an absurd amount of 
hype behind them, and a lot of people don't fully understand what they are. 

In order for me to describe what containers are, let's first establish what
virtual machines are, and why they exist. Before any of this container or VM
stuff came about, you would just run applications straight on your server. 
If Sally wanted PHP 5.6 on the server to run her stuff, and Bob wanted PHP 5.1
to run something else, they'd either need to complicate their server by having multiple versions of PHP available, or Bob would need to run his stuff on a different machine.

This is annoying, and it's a problem that VMs can solve. Instead of getting a
whole new computer to run his shitty little website, we have software built to
help you put a secondary OS on the same machine through lies. The new OS
\textit{thinks} it's on its own hardware, but all of the hardware bits that
it is twiddling are secretly just emulated by the original OS. The virtual
machine's kernel goes to send a packet to the network interface? Nah, it's 
just a thing in software that goes to the host OS. The kernel tries to write a
file? More lies. That's not a real hard drive, but just a file on the host OS. 
Now Sally can run her new and shiny PHP app, and Bob can run his old and shitty
PHP app, and each of them get their own web server with only the version of PHP
that they want.

Many companies have built fortunes using lies - err, VMs - to be able to give
each little service / application its own machine. One annoying thing about
this strategy though is it can take a significant amount of work to set up a
VM. If you've ever installed Linux, imagine going through that process manually
for each set of things you want to run. There is tooling around doing this
automagically for you, but if you're not paying Amazon you'll be setting that
up yourself.

In our hypothetical scenario where Bob insists on using a version of PHP from
2005 for some god awful reason, what if we could streamline the process by
using significantly less lies? Ultimately the end goal is just having multiple
web servers with different versions of PHP running on the same machine. Do we
really need to emulate network cards, CPUs, hard drives, and more for each web
server? What if we could only lie about the versions of PHP available on the
system, and let the different web servers use the real block devices and memory
space on the host?

Containers are, in a sense, exactly that. Utilizing some features in the Linux
kernel, we can just lie to a process about the following things:
\begin{itemize}
\item Where `/` is in the filesystem. This means that when a process tries to load
  in `/usr/lib/php5`, what it really gets is
  `/var/lib/rkt/stuff/rootfs/usr/lib/php5`.
\item What other processes are running on the system. This means that the two web
  servers won't be able to see - and therefore interfere - with each other (or
  much else for that matter).
\item The network stack. It can be useful to (optionally) hand each web server a
  separate network stack, that is separate from the host's. This means that
  both web servers can bind to port 80, and when the host gets a connection to
  port 80 it'll just forward the connection to one of the two web servers,
  depending on what application the user is trying to access.
\end{itemize}

So there you have it. Containers are kinda like VMs in that they're also built
with the goal of lying to parts of your computer, but they aim to be easier to
use/setup by lying about significantly less.

%-----------------------------------------------------------

\end{minipage} % End the main body - first page mini page

\end{document} 
