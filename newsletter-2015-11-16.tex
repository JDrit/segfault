%%%%%%%%%%%%%%%%%%%%%%%%%%%%%%%%%%%%%%%%%
% Professional Newsletter Template
% LaTeX Template
% Version 1.0 (09/03/14)
%
% Created by:
% Bob Kerstetter (https://www.tug.org/texshowcase/) and extensively modified by:
% Vel (vel@latextemplates.com)
% 
% This template has been downloaded from:
% http://www.LaTeXTemplates.com
%
% License:
% CC BY-NC-SA 3.0 (http://creativecommons.org/licenses/by-nc-sa/3.0/)
%
%%%%%%%%%%%%%%%%%%%%%%%%%%%%%%%%%%%%%%%%%

\documentclass[9pt]{extarticle} % The default font size is 10pt; 11pt and 12pt are alternatives
\usepackage[procnames]{listings}
\usepackage{color}
\usepackage{makecell}

\input{structure.tex} % Include the document which specifies all packages and structural customizations for this template

\begin{document}

\definecolor{keywords}{RGB}{255,0,90}
\definecolor{comments}{RGB}{0,0,113}
\definecolor{red}{RGB}{160,0,0}
\definecolor{green}{RGB}{0,150,0}
 
\lstset{language=Python, 
        basicstyle=\ttfamily\small, 
        keywordstyle=\color{keywords},
        commentstyle=\color{comments},
        showstringspaces=false,
        procnamekeys={def,class}}

%--------------------------------------------------------------------------------
% HEADER DETAILS
%--------------------------------------------------------------------------------

\pagestyle{fancy}
\fancyhf{}
\chead{segfault@csh.rit.edu}
\rhead{\today}
\lhead{Volume XLVII Issue \#{8}}
\addtolength\footskip{-15px}
\cfoot{"Do you know how hard it is to embezzle money when there is no money?" -Marc billow (mbillow)}

%----------------------------------------------------------------------------------------
%	HEADER IMAGE
%----------------------------------------------------------------------------------------

\begin{figure}[H]
\centering\vspace{0.5cm}\includegraphics[width=0.8\linewidth]{segfault.png}
\end{figure}

%--------------------------------------------------------------------------------
% HEADER QUOTE
%--------------------------------------------------------------------------------

\vspace{-15px}
\begin{quote}
\centering
\textbf{\textit{Why do today if you can do tomorrow?}}
\end{quote}
\vspace{10px}

%----------------------------------------------------------------------------------------
%	SIDEBAR - FIRST PAGE
%----------------------------------------------------------------------------------------

\vspace{-0.5cm}\begin{minipage}[t]{.30\linewidth} % Mini page taking up 30% of the actual page
\begin{mdframed}[style=sidebar,frametitle={}] % Sidebar box

%-----------------------------------------------------------

\hypertarget{contents}{\textbf{{\large This week on floor\ldots}}} % \hypertarget provides a label to reference using \hyperlink{label}{link text}
\begin{itemize}
\item \hyperlink{firstnews}{What is Tail Recursion?}
\item \hyperlink{secondnews}{Death by the Deaf}
\item \hyperlink{thirdnews}{Employer Reviews}
\end{itemize}

\centerline {\rule{.75\linewidth}{.25pt}} % Horizontal line

%-----------------------------------------------------------

\textbf{Notable Upcoming Events:}
\begin{enumerate}[leftmargin=0.2cm]
\item \textbf{Potter Seminar} Nov. 19, 7pm \\
	Because you can always learn more about networking
\\
\item \textbf{Rocky Horror Show} Nov. 20, 12pm\\
	CSH is going to go see the show run by the Collage of
	Liberal Arts. It is located	on campus in the LBJ building. 
	Cost is free thanks to our amazing RA Kevin Dolan!
\\
\item \textbf{Mini Maker Fair} Nov. 21 \\
	Come support your fellow CSHers at the maker fair!
\\
\item \textbf{Jam Your Way To GDC} Nov. 21, 5pm - Nov. 22, 5pm\\
	Aidan McInerny is hosting a Game Jam with major prizes.
	Come show off your skills for the chance to win big!
\\ 
\item \textbf{Hipster Movie Night} Nov. 21, 10pm \\
	Omar is hosting a movie night! Come see a new movie
	that you haven't seen before. Guaranteed not to be
	disappointed.
\\
\end{enumerate}

%-----------------------------------------------------------


%-----------------------------------------------------------

\captionof*{table}{Voting Results}
\begin{tabular}{lcr}

Vote & Cost & Result \\
\midrule
\bottomrule
\end{tabular}

%-----------------------------------------------------------

\end{mdframed}
\end{minipage}\hfill % End the sidebar mini page 
%
%----------------------------------------------------------------------------------------
%	MAIN BODY - FIRST PAGE
%----------------------------------------------------------------------------------------
%
\begin{minipage}[t]{.66\linewidth} % Mini page taking up 66% of the actual page
\vspace{-0.4cm}
\hypertarget{firstnews}{\heading{What is Tail Recursion?}{6pt}}
	A function call is said to be tail recursive if there is nothing to do after the function returns except return its value. Tail recursion is important because it can be implemented more efficiently than general recursion. The compiler will normally just us a \textbf{goto} under the hood. When we make a normal recursive call, we have to push the return address onto the call stack then jump to the called function. This means that we need a call stack whose size is linear in the depth of the recursive calls. When we use tail recursion we know that as soon as we return from the recursive call we are going to immediately return as well, so we can skip the entire chain of recursive functions returning and return straight to the original caller. That means we don't need a call stack at all for all of the	recursive calls, and can implement the final call as a simple jump, which saves us space. \\
\\
Below is the same algorithm written using general recursion (left) and using tail recursion (right). While both of these algorithms produce the same result, they have different call stacks. \\

\begin{tabular}{ l c r }
\begin{lstlisting}
def recsum(x):
  if x == 1:
    return x
  else:
    return x + recsum(x-1)
\end{lstlisting}
&
\begin{lstlisting}
def tailrecsum(x, accum=0):
  if x == 0:
    return accum
  else:
    return tailrecsum(x-1, accum+x)
\end{lstlisting}
\\
\\
\makecell{
recsum(5) \\
5 + recsum(4) \\
5 + (4 + recsum(3)) \\
5 + (4 + (3 + recsum(2))) \\
5 + (4 + (3 + (2 + recsum(1)))) \\
5 + (4 + (3 + (2 + 1))) \\
15 \\}

&
\makecell{
tailrecsum(5, 0) \\
tailrecsum(4, 5) \\
tailrecsum(3, 9) \\
tailrecsum(2, 12) \\
tailrecsum(1, 14) \\
tailrecsum(0, 15) \\
15 \\}
\end{tabular}

The call stack on the left requires you to go up the call stack to finish
adding all the numbers together while the call stack on the right just
needs to return the final solution.

%-----------------------------------------------------------

\hypertarget{secondnews}{\heading{Death by the Deaf}{6pt}} 

During Death by Dagger, fall of 2007, freshman Matt Olivo was traveling 
through the tunnels when he bumped into Kevin Pierce, who was his target.
Matt quickly killed Kevin. Incidentally there were a group of students walking
in front of them. Kevin started yelling at them for them to witness the kill
until he realized that they were all deaf. Which at that moment, Matt said 
"they didn't here shit." So remember kids, always make sure that your witness
actually see the kill.

\hypertarget{thirdnews}{\heading{Employer Reviews}{6pt}} 

It is that time of year again when everyone is going out and interviewing for 
jobs. Some of us even just returned from summer co-ops. For all people returning from co-ops, please take some time and write up an employee review on CSH's 
wiki page (link below). It is really great that we have so many members going
to interesting companies and it would be really useful to learn your thoughts
about your previous employees. It is also a great way for freshmen to learn 
what to expect from different companies. Thanks for helping CSH be!
\centerline{\url{https://wiki.csh.rit.edu/wiki/Employer_Reviews}}

\end{minipage} % End the main body - first page mini page

\end{document} 