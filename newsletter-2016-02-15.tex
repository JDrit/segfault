%%%%%%%%%%%%%%%%%%%%%%%%%%%%%%%%%%%%%%%%%
% Professional Newsletter Template
% LaTeX Template
% Version 1.0 (09/03/14)
%
% Created by:
% Bob Kerstetter (https://www.tug.org/texshowcase/) and extensively modified by:
% Vel (vel@latextemplates.com)
% 
% This template has been downloaded from:
% http://www.LaTeXTemplates.com
%
% License:
% CC BY-NC-SA 3.0 (http://creativecommons.org/licenses/by-nc-sa/3.0/)
%
%%%%%%%%%%%%%%%%%%%%%%%%%%%%%%%%%%%%%%%%%

\documentclass[9pt]{extarticle} % The default font size is 10pt; 11pt and 12pt are alternatives

\input{structure.tex} % Include the document which specifies all packages and structural customizations for this template

\usepackage{amssymb}% http://ctan.org/pkg/amssymb
\usepackage{proof}
\usepackage{pifont}% http://ctan.org/pkg/pifont
\newcommand{\cmark}{\ding{51}}%
\newcommand{\xmark}{\ding{55}}%

\begin{document}

%--------------------------------------------------------------------------------
% HEADER DETAILS
%--------------------------------------------------------------------------------

\pagestyle{fancy}
\fancyhf{}
\chead{segfault@csh.rit.edu}
\rhead{\today}
\lhead{Volume XLVIII Issue \#4}
\addtolength\footskip{-15px}
\cfoot{This [porn] has too much plot" - Henry Saniuk (henry)}

%----------------------------------------------------------------------------------------
%	HEADER IMAGE
%----------------------------------------------------------------------------------------

\begin{figure}[H]
\centering\vspace{0.5cm}\includegraphics[width=0.8\linewidth]{imgs/segfault.png}
\end{figure}

%--------------------------------------------------------------------------------
% HEADER QUOTE
%--------------------------------------------------------------------------------

\vspace{-15px}
\begin{quote}
\centering
\textbf{\textit{When your hammer is C++, everything begins to look like a thumb}}
\end{quote}
\vspace{10px}

%----------------------------------------------------------------------------------------
%	SIDEBAR - FIRST PAGE
%----------------------------------------------------------------------------------------

\vspace{-0.5cm}\begin{minipage}[t]{.44\linewidth} % Mini page taking up 35% of the actual page
\begin{mdframed}[style=sidebar,frametitle={}] % Sidebar box

%-----------------------------------------------------------

\hypertarget{contents}{\textbf{{\large This week on floor\ldots}}} % \hypertarget provides a label to reference using \hyperlink{label}{link text}
\begin{itemize}
\item \hyperlink{firstnews}{Subtyping}
\item \hyperlink{secondnews}{Making things for CSHers Response}
\end{itemize}

\centerline {\rule{.75\linewidth}{.25pt}} % Horizontal line

%-----------------------------------------------------------

\textbf{Notable Upcoming Events:}
\begin{enumerate}[leftmargin=0.2cm]
\item \textbf{How Gates Work Seminar} Feb. 16, 7pm \\
	Jello will be giving a seminar about Gates. Gates was an American computer programmer, who alongside Paul Allen, founded Microsoft in 1975. Since then he has been an influential part of the company as it has grown throughout time. Come learn more about Gates with Jello.
\\
\item \textbf{GCCIS Fair: Who's In Our House?} Feb. 17, 3pm \\
	GCCIS is having a club fair for CS related organizations, located in the atrium. Try and come support CSH if you are free during this time. Talk to Victoria if you have any additional questions.
\\
\item \textbf{Boyer-Moore String Searching Seminar} Feb. 17, 9pm \\
	Slackwill is giving a seminar about a very efficient algorithm to do string searching. It is really cool, and I am not even saying that since I am obligated to. Why don't you come learn a thing or two?
\\
\item \textbf{War Paint Challenge} Feb. 19 \\
	CSH will be hosting the War Paint Challenge, alongside several other SIHs. Come eat wings or cheer on your fellow house mates!
\\
\item \textbf{Drag Night} Feb. 19, 6pm \\
	Derek is hosting drag night once again! There will be a shopping run for stuff beforehand so keep up to date with news for more information.
\\
\item \textbf{Freezefest} Feb. 20 \\
	With John Mulaney and Nick Kroll, located in the Gordon Field House.
\\
\item \textbf{Remote Hackathon Presentations} Feb. 21, 5pm \\
	Come see all the cool projects CSHers worked on over break!!
\end{enumerate}

%-----------------------------------------------------------


\textbf{Joke of the Week} \\
I would tell you a UDP joke, but you might not get it.
\\
\\
%-----------------------------------------------------------
\textbf{Constitutional Amendments} \\ 
\textbf{Proposed}: There is currently a proposal out to change the voting for Spring evaluations. Currently it is a secret ballot in which we all play heads-up 7up, and this amendment will make it an open vote. \\
\\
\textbf{Voting}: The current amendment being voted on is if we should allow an easier process to submit non-semantic changes to the constitution. This was brought up by Matt Gambogi.


%-----------------------------------------------------------

\captionof*{table}{Voting Results}
\begin{tabular}{lcr}

Vote & Cost & Result \\
\midrule
Pint Glasses & \$350 & PASSED \\
Laser Printer & \$400 & \textbf{FAILED} \\
OpComm Shit & \$1020 & PASSED \\
40th Money & \$20,000 & PASSED \\
\bottomrule
\end{tabular}

%-----------------------------------------------------------

\end{mdframed}
\end{minipage}\hfill % End the sidebar mini page 
%
%----------------------------------------------------------------------------------------
%	MAIN BODY - FIRST PAGE
%----------------------------------------------------------------------------------------
%
\begin{minipage}[t]{.52\linewidth} % Mini page taking up 61% of the actual page
\vspace{-0.4cm}
\hypertarget{firstnews}{\heading{Subtyping}{6pt}}

The concept of subtyping is found throughout most programming languages. What it
entails is that if we are given type \textit{A}, then we can defined a narrower
type \textit{B}, such that all values of type \textit{B} are also of type 
\textit{A}. This relationship is written as \textit{B} <: \textit{A}. 
This subtyping relationship is reflexive (\textit{A} <: \textit{A})  and transitive
(\textit{A} <: \textit{B} $\bigwedge$ \textit{B} <: \textit{C} $\rightarrow$
\textit{A} <: \textit{C}). Function subtyping is defined as: \\
\\
\centerline{$\infer{\tau'_a\ \rightarrow\ \tau'_r\ <:\ \tau_a\ \rightarrow\ \tau_r}
	   {\tau_a\ <:\ \tau'_a\ \ \ \tau'_r\ <:\ \tau_r}$}
\\
\textbf{Covariance}: \\
Say for example we have a function of type \textit{Boolean} $\rightarrow$
\textit{Integer} and that \textit{Integer} <: \textit{Double}. This return type
of this function is \textit{Integer} though it can be also be subtyped to return 
types of \textit{Double}. This \textbf{replacement of wider types with narrower
types} is called covariance. Function are covariant in their result types. \\
\\
\centerline{$\infer{boolean\ \rightarrow\ integer\ <:\ boolean \rightarrow\ double}
	   {boolean\ <:\ boolean\ \ \ integer\ <:\ double}$}
\\
\textbf{Contravariance}: \\
Consider the function of type \textit{Double} $\rightarrow$ \textit{Boolean}. We
can use this function anywhere that expects a function of type \textit{integer}
$\rightarrow$ \textit{boolean}. This \textbf{replacement of narrower types with 
wider types} is called contravirance. This is allowed since \textit{integer} <:
\textit{double}.Functions are contravariant in their argument types. \\
\\
\centerline{$\infer{double \rightarrow\ boolean\ <:\ integer\ \rightarrow\ boolean}
	   {integer\ <:\ double\ \ \ boolean\ <:\ boolean}$}
\\
\\
\textbf{Lambda Calculus Example}: \\
$\lambda$ x. \{$l_1$: int, $l_2$: int\}. \{$l_1$: x.$l_2$, $l_2$: x.$l_1$\} \\
$\cdot$ $\vdash$ $\lambda$ x. \{$l_1$: int, $l_2$: int\}. \{$l_1$: x.$l_2$, $l_2$: x.$l_1$\} : \{$l_1$: int, $l_2$: int\} $\rightarrow$ \{$l_1$: int, $l_2$: int\}  \\
\\
S-RCDWIDTH: 
$\infer{\{l_1\ :\ \tau_1,\ ...,\ l_{n+1}\ :\ \tau_{n+1}\}\ <:\ \{l_1\ :\ \tau_1,\ ...,\ l_n\ :\ \tau_n\}}{}$ \\
\\
While this seems complicated, all it does is define a function that takes a record
of size 2 and returns the input record with the elements swapped in it. The type 
rule S-RCDWIDTH gives us the declaration of width subtyping. This rule just says
that records with more fields is a subtype of records with smaller amount of fields,
if all the other fields left are the same type. This function can be subtyped to 
many different types as long as we are contravariance in the argument type and covariance in the result. 

\cmark \{$l_1$: int, $l_2$: int, $l_3$: int\} $\rightarrow$ \{$l_1$: int\} \\
\xmark \{$l_1$: int\} $\rightarrow$ \{$l_1$: int\} \\
\xmark \{$l_1$: int, $l_2$: int\} $\rightarrow$ \{$l_1$: int, $l_2$: int\, $l_3$: int\}  \\
The first one type checks since it is contraviariant in the argument type and
covariant in the result. This is valid since
\{$l_1$: int, $l_2$: int, $l_3$: int\} <: \{$l_1$: int, $l_2$: int\} and 
\{$l_1$: int, $l_2$: int\} <: \{$l_1$: int\}. The other two do not follow the
typing rule S-RCDWIDTH. 

\end{minipage} % End the main body - first page mini page
\newpage

%----------------------------------------------------------------------------------------
%	HEADER IMAGE
%----------------------------------------------------------------------------------------

\begin{figure}[H]
\centering\vspace{0.5cm}\includegraphics[width=0.8\linewidth]{imgs/segfault.png}
\end{figure}

%--------------------------------------------------------------------------------
% HEADER QUOTE
%--------------------------------------------------------------------------------

\vspace{-15px}
\begin{quote}
\centering
\textbf{\textit{When your hammer is C++, everything begins to look like a thumb}}
\end{quote}
\vspace{5px}
\hypertarget{secondnews}{\heading{Making things for CSHers Response}{6pt}}


This is in response to the "Making things for humans" article posted last week. Those are great guidelines for making things for mass market, but CSH is not the mass market and thus you should design accordingly. 
\\
\\
My biggest complaint about the guidelines posted is they prioritize making things simple over making things that are customizable. Customizability is not evil; it means every user can tweak your project to meet their use case. Don't strip away useful features in an effort to make your UI simpler. If you force people to use your project in a particular way, people will only ever use it in that way. One of the great guiding principles of the Unix philosophy is to write programs that can work together, even if you didn't know what the other programs would be. Try to build your project so it can be used as generally as possible build it with reuse in mind. Maybe someone will use it as part of their project.
\\
\\
Don't hide the inner workings of your project. A lot of programs are only "confusing"
because they hide what they're doing. Both Windows and OSX hide the underlying file structure from you (ever try to go up a level from your user directory in finder or explorer?) and as a result people never learn about how their system is laid out and find it confusing.
\\
\\
Most importantly, build what you want to build it's your project. Don't fret over UI if you don't care about UI, but if that's your jam, go ham on it. You won't satisfy everyone, so you might as well satisfy yourself. Drew and I have very different opinions on what an interface should consist of, and you probably do too. Just don't take either side as gospel, it's as futile as the Vim vs Emacs wars.
\\
\\
\\
\includegraphics[width=\linewidth]{imgs/cstheory.png} 
\end{document} 
