%%%%%%%%%%%%%%%%%%%%%%%%%%%%%%%%%%%%%%%%%
% Professional Newsletter Template
% LaTeX Template
% Version 1.0 (09/03/14)
%
% Created by:
% Bob Kerstetter (https://www.tug.org/texshowcase/) and extensively modified by:
% Vel (vel@latextemplates.com)
% 
% This template has been downloaded from:
% http://www.LaTeXTemplates.com
%
% License:
% CC BY-NC-SA 3.0 (http://creativecommons.org/licenses/by-nc-sa/3.0/)
%
%%%%%%%%%%%%%%%%%%%%%%%%%%%%%%%%%%%%%%%%%

\documentclass[9pt]{extarticle} % The default font size is 10pt; 11pt and 12pt are alternatives

\input{structure.tex} % Include the document which specifies all packages and structural customizations for this template

\begin{document}

%--------------------------------------------------------------------------------
% HEADER DETAILS
%--------------------------------------------------------------------------------

\pagestyle{fancy}
\fancyhf{}
\chead{segfault@csh.rit.edu}
\rhead{\today}
\lhead{Volume XLVIII Issue \#9}
\addtolength\footskip{-15px}
\cfoot{<QUOTE>}

%----------------------------------------------------------------------------------------
%	HEADER IMAGE
%----------------------------------------------------------------------------------------

\begin{figure}[H]
\centering\vspace{0.5cm}\includegraphics[width=0.8\linewidth]{imgs/segfault.png}
\end{figure}

%--------------------------------------------------------------------------------
% HEADER QUOTE
%--------------------------------------------------------------------------------

\vspace{-15px}
\begin{quote}
\centering
\textbf{\textit{<SUBTITLE>}}
\end{quote}
\vspace{10px}

%----------------------------------------------------------------------------------------
%	SIDEBAR - FIRST PAGE
%----------------------------------------------------------------------------------------

\vspace{-0.5cm}\begin{minipage}[t]{.35\linewidth} % Mini page taking up 35% of the actual page
\begin{mdframed}[style=sidebar,frametitle={}] % Sidebar box

%-----------------------------------------------------------

\hypertarget{contents}{\textbf{{\large This week on floor\ldots}}} % \hypertarget provides a label to reference using \hyperlink{label}{link text}
\begin{itemize}
\item \hyperlink{firstnews}{B-Trees}
\item \hyperlink{secondnews}{Software Engineer Translator}
\end{itemize}

\centerline {\rule{.75\linewidth}{.25pt}} % Horizontal line

%-----------------------------------------------------------

\textbf{Notable Upcoming Events:}
\begin{enumerate}[leftmargin=0.2cm]
\item \textbf{<EVENT NAME>} <DATE + TIME> \\
	<DESCRIPTION>
\\
\item \textbf{<EVENT NAME>} <DATE + TIME> \\
	<DESCRIPTION>
\\
\end{enumerate}

%-----------------------------------------------------------


\textbf{<TITLE>} \\
<SHORT SECTION>
\\

%-----------------------------------------------------------

\captionof*{table}{Voting Results}
\begin{tabular}{lcr}

Vote & Cost & Result \\
\midrule
<NAME> & \$<MONEY> & <STATUS> \\
\bottomrule
\end{tabular}

%-----------------------------------------------------------

\end{mdframed}
\end{minipage}\hfill % End the sidebar mini page 
%
%----------------------------------------------------------------------------------------
%	MAIN BODY - FIRST PAGE
%----------------------------------------------------------------------------------------
%
\begin{minipage}[t]{.61\linewidth} % Mini page taking up 61% of the actual page
\vspace{-0.4cm}
\hypertarget{firstnews}{\heading{B-Trees}{6pt}}
 
A b-tree is a type of self-balancing tree data structure that is heavily used in databases and
compute clusters. It is a type of binary search tree but it differs in that a node can have more
than 2 children. In fact, it has as many children as it can so that they all fit into a block on
disk. This way all of the children are read into memory with only 1 read. This makes it so that
they are optimized for systems that read and write large blocks of data, aka databases.

They wre introduced in 1907, and are still found in almost all major system today. They remain
to this date the standard way to implement on-disk indexes for almost all major relational
databases. Traditionally b-trees break each node into blocks of 4KB of size, due to the average
block size on-disk. Most of the data is stored on disk so that it can handle more than the size
of the memory on the machine. It takes advantage of the system's page cache to keep the most use
nodes in memory.

\includegraphics[width=0.8\linewidth]{imgs/b-tree.png}

One page is designated as the root node for the b-tree. The first step is to pull that node from
disk. It is normally located in memory already so the cost is very cheap. This node contains k
children references and the range of nodes that each is responsible for. You determine which
child you want to visit and repeat the process with reading it from disk. This process is designed
to do as little disk IO as possible.

%-----------------------------------------------------------

\hypertarget{secondnews}{\heading{Software Engineer Translator}{6pt}}

\textbf{What is said:} This is a horrible hack \\
\textbf{What it means:} This is a something that I did not write

\textbf{What is said:} It's broken \\
\textbf{What it means:} There are bugs in your code

\textbf{What is said:} It has a few issues \\
\textbf{What it means:} There are bugs in my code

\textbf{What is said:} It is self-documenting \\
\textbf{What it means:} My code doesn't have comments

\textbf{What is said:} emacs is better than vim \\
\textbf{What it means:} It's too peaceful, let's start a flame war

\textbf{What is said:} vim is better than emacs \\
\textbf{What it means:} It's too peaceful, let's start a flame war

\textbf{What is said:} Legacy code \\
\textbf{What it means:} It works, but no ones knows how




\end{minipage} % End the main body - first page mini page

\end{document} 
