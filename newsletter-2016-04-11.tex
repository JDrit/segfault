%%%%%%%%%%%%%%%%%%%%%%%%%%%%%%%%%%%%%%%%%
% Professional Newsletter Template
% LaTeX Template
% Version 1.0 (09/03/14)
%
% Created by:
% Bob Kerstetter (https://www.tug.org/texshowcase/) and extensively modified by:
% Vel (vel@latextemplates.com)
% 
% This template has been downloaded from:
% http://www.LaTeXTemplates.com
%
% License:
% CC BY-NC-SA 3.0 (http://creativecommons.org/licenses/by-nc-sa/3.0/)
%
%%%%%%%%%%%%%%%%%%%%%%%%%%%%%%%%%%%%%%%%%

\documentclass[9pt]{extarticle} % The default font size is 10pt; 11pt and 12pt are alternatives

\input{structure.tex} % Include the document which specifies all packages and structural customizations for this template

\begin{document}

%--------------------------------------------------------------------------------
% HEADER DETAILS
%--------------------------------------------------------------------------------

\pagestyle{fancy}
\fancyhf{}
\chead{segfault@csh.rit.edu}
\rhead{\today}
\lhead{Volume XLVIII Issue \#8}
\addtolength\footskip{-15px}
\cfoot{"I'm pretty sure men can have female orgams, and that is pretty great!" - John King (johnsk)}

%----------------------------------------------------------------------------------------
%	HEADER IMAGE
%----------------------------------------------------------------------------------------

\begin{figure}[H]
\centering\vspace{0.5cm}\includegraphics[width=0.8\linewidth]{imgs/segfault.png}
\end{figure}

%--------------------------------------------------------------------------------
% HEADER QUOTE
%--------------------------------------------------------------------------------

\vspace{-15px}
\begin{quote}
\centering
\textbf{\textit{ALMOST TIME FOR 40th!!!}}
\end{quote}
\vspace{10px}

%----------------------------------------------------------------------------------------
%	SIDEBAR - FIRST PAGE
%----------------------------------------------------------------------------------------

\vspace{-0.5cm}\begin{minipage}[t]{.35\linewidth} % Mini page taking up 35% of the actual page
\begin{mdframed}[style=sidebar,frametitle={}] % Sidebar box

%-----------------------------------------------------------

\hypertarget{contents}{\textbf{{\large This week on floor\ldots}}} % \hypertarget provides a label to reference using \hyperlink{label}{link text}
\begin{itemize}
\item \hyperlink{firstnews}{The superior design of the station wagon}
\item \hyperlink{secondnews}{Classes of Problems}
\end{itemize}

\centerline {\rule{.75\linewidth}{.25pt}} % Horizontal line

%-----------------------------------------------------------

\textbf{Notable Upcoming Events:}
\begin{enumerate}[leftmargin=0.2cm]
\item \textbf{WITR Talk} Apr. 13, 9pm \\
  \\
\item \textbf{Super Game Jam Showing} Apr. 13, 10pm \\
  Zach and Noah are
  \\
\item \textbf{40th!!!} Apr. 15 - 17 \\
  The event you have all been waiting for is finally here! Hello to all the
  alumni that are reading this.
  \\
\item \textbf{Polisseni Tour} Apr. 16, 2pm \\
  There will be tours going on of RIT new hockey arena for all members, more
  information to come.
  \\
\item \textbf{Formal Dinner} Apr. 16, 6pm \\
    In the evening there will be a formal banquet held in the Davis Room of the
    Student Alumni Union. There will be plenty of refreshments, a dance floor,
    and a brief speech on the current state of Computer Science House.
    \\
\end{enumerate}

%-----------------------------------------------------------

\textbf{Amendments:} \\
\textbf{optional dicussion:} Max's proposal is to move all discussions for amendments
to the end of house meeting and make them optional for all members. Any one who
leaves will have their vote counted as an abstention.
\\

%-----------------------------------------------------------

\captionof*{table}{Voting Results}
\begin{tabular}{lcr}

Vote & Cost & Result \\
\midrule
<NAME> & \$<MONEY> & <STATUS> \\
\bottomrule
\end{tabular}

%-----------------------------------------------------------

\end{mdframed}
\end{minipage}\hfill % End the sidebar mini page 
%
%----------------------------------------------------------------------------------------
%	MAIN BODY - FIRST PAGE
%----------------------------------------------------------------------------------------
%
\begin{minipage}[t]{.61\linewidth} % Mini page taking up 61% of the actual page
\vspace{-0.4cm}
\hypertarget{firstnews}{\heading{The superior design of the station wagon}{6pt}}

Over the past couple decades we have witnessed the slow death of the station wagon. At the same time we have seen the rise of the crossover which emerged to fill the void the wagon left. There are many differing opinions on why this happened, but regardless of reason, it is a tragedy. Station wagons offer the best of both sedans and SUVs while crossovers are the bastard children from when a minivan got freaky with a sedan.

Station wagons are the superior car choice. They are much lower to the ground and as a result offer better handling. The station wagon will gracefully slide around corners while the crossover will flip and roll. The wagon's low, sleek profile also reduces aerodynamic drag, resulting in better fuel economy and tighter handling at high speed. 

Not only do station wagons drive better, but they are more practical as well. They offer more usable cargo space and roof space. Ever tried to fit 2x4's or a full size door in a crossover? The station wagon eats 2x4's and doors for breakfast, then comes back for a second helping of plywood. While crossovers typically have more vertical store, I would argue the extra length is significantly more useful. Their longer cargo space also better accommodates humans. Fold down the seats and you have the ideal space to sleep, relax, or do other human activities.

%-----------------------------------------------------------

\hypertarget{secondnews}{\heading{Classes of Problems}{6pt}} 

We will be going over different classes of decision problems and how they relate to each
other. Remember that all that decision problems are yes or no answers to a given input.

\textbf{P: Polynomial Time} \\
P is the complexity class for all problems that are solvable by a deterministic turning machine in
time polynomial to the size of the input. This includes algorithms that have Big-O such as O(n),
O(log(n)), and even O($n^{100}$).

\textbf{NP: Non-deterministic Polynomial Time} \\
NP is the class of problems that are harder to solve for. There does not exists a polynomial time
deterministic turning machine to solve these problems. While they are hard to solve for, if given an
answer, it can be verified in polynomial time. Integer factorisation is a NP problem.

\textbf{NP-Complete: Non-deterministic Polynomial Complete Time} \\
NP-Complete is a special subset of NP algorithms. What makes these special is that all other
problems in NP are polynomial time, many-to-one reducible to every NP-Complete problem. This is
important since it we are able to find a polynomial time solution to \textbf{any} NP-Complete
problem, then we will be able to solve any NP problem in polynomial time. The boolean satisfiability
problem is an example of an NP-Complete problem.

\textbf{NP-Hard: Non-deterministic Polynomial Hard Time} \\
NP-Hard problems are problems that are \textit{at least as hard as} NP-Complete problems. What this
means is that a problem is NP-Hard if there is an NP-Complete problem that is reducible to it in
polynomial time. An example of an NP-Hard problem is the Halting problem.


\end{minipage} % End the main body - first page mini page

\end{document} 
